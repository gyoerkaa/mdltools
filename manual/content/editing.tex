\chapter{Editing}
There are three types of objects which can be exported to a MDL: Empties,
Meshes and Lamps. Other objects like curves and surfaces will have to be
converted to meshes before attempting to export them.

\section{Empties}
Empties are used to group objects and indicate special 
locations. Empties are called Dummies in MDL files.


\subsection{Aurora Root}
Each MDL requires at least one Empty: The Aurora Root or Rootdummy. It holds 
additional information and indicates which objects belong to an MDL. All children
of a single Rootdummy are considered to be part of the same MDL. \\

Any Empty without a parent is automatically considered a Rootdummy. 
The Rootdummy must have the same name as the MDL file (minus the file extension). 
A Rootdummy has the following properties:

\begin{propertyAurora}{Classification}
Determines the type of MDL
\begin{description}[leftmargin=6em,style=nextline]
    \item[Item] Any inventory item.
    \item[GUI] Unknown.
    \item[Effect] For visual effects.
    \item[Door] Doors, be it generic or for tilesets.
    \item[Character] Creature or placeables. 
    \item[Tile] For tilesets.
    \item[Unknown] Avoid using this. It may be set during import.
\end{description}
\end{propertyAurora}

\begin{propertyAurora}{Supermodel}
Reference to another MDL file. All animations from that file 
will be available in this MDL.
\end{propertyAurora}

\begin{propertyAurora}{Animationscale}
Reference to another MDL file. All animations from that file 
will be available in this MDL.
\end{propertyAurora}

\subsection{Dummy}
Simple Empties with a parent are either used to group objects or
indicate a special location to the engine. \\

\begin{figure}
  \centering
  \includegraphics[trim=0 0 0 0, clip, width=0.33\textwidth]{panel_dummy}
  \caption[panel dummy]{Dummy Property Panel}
  \label{fig:panel_dummy}
\end{figure}

\begin{propertyAurora}{Dummytype}
The following types of Dummies are available:
\begin{description}[leftmargin=6em,style=nextline]
    \item[None] Simple dummy without any special purpose.
    \item[Ground] Indicates the ground level for spells/ visual effects
    \item[Impact] Impact target for most spells/ visual effects.
    \item[Head Hit] Spells/ visual effects
    \item[Head] Head location for Spells/ visual effects
    \item[Hand] Point of origin for (some) spells. If a spell/ visual is originating from this mdl, this will be the point of origin most of the time.
    \item[Use 1] Use Node for placeables or doors. Upon receiving a use command a character will move to the closest of the two use nodes.
    \item[Use 2] Use Node for placeables or doors. Upon receiving a use command a character will move to the closest of the two use nodes.
\end{description}
\end{propertyAurora}
None of the special Dummies are mandatory. If one is
missing from the MDL, the engine will use (0,0,0) as the default location
for that particular Dummy. \\

Selecting the type is not enough to make a Dummy work, they need to have
the correct name or to be precise: The correct suffix.
The export script will try to generate a valid name depending on
the Dummy's type, but it is recommended to do it manually to avoid naming
conflicts. \\

Use the button next to the Dummytype selector to generate a
working name. The Rootdummy's name will be used to create a working
suffix for the selected Dummytype. \\

\subsection{Reference Node}
The purpose and usage of these nodes is unknown. Supposedly it can be used to
reference other MDL files. These nodes can be found in some visual effects and reference fx\_ref 
most of the time.

\begin{propertyAurora}{Reference Model}
The name of another MDL file without file-extension.
\end{propertyAurora}

\begin{propertyAurora}{Reattachable}
Unknown purpose.
\end{propertyAurora}

\subsection{Patch Node}
The purpose and usage of these nodes is unknown. They occur in some visual effects, but
they seem to behave exactly like normal dummy nodes, i.e they have the same
attributes.

\section{Meshes}

\subsection{Trimeshes}
Any newly created mesh object is a trimesh by default, the most common type of mesh 
used by the aurora engine. As the name suggests it consists of triangles, but it is 
not necessary to explicitly convert the faces to triangles as Neverblender does that 
automatically during the export process.

\begin{propertyBlender}{Scale}
Blender's scale factor property cannot be exported directly as the engine 
only supports uniform scale. If a non-uniform scale value is used the export 
script will try to apply the scaling factor to the object and it's children.
This may cause issues with animation. Therefore it is recommended to either keep 
the scale uniform or apply it manually before exporting.
\end{propertyBlender}  
 
\begin{propertyAurora}{Wirecolor}
This property is unused by the engine. It can safely be ignored. It probably defines the
color for the object's wire-frame in 3dsmax and exists in Neverblender only for 
compatibility reasons.
\end{propertyAurora}

\begin{propertyAurora}{Self-Illumination Color}
Makes the mesh seem to glow. It does not act as a light source however.
This property will not be visible in Blender.
\end{propertyAurora}

\begin{propertyAurora}{Ambient Color}
OpenGL material property. The engine uses per-object ambient light whereas Blender's 
defines it on a per scene basis. Therefore this property will not be visible in Blender.
\end{propertyAurora}

\begin{propertyAurora}{Shininess}
Shininess requires a matching texture information file (*.txi).
\end{propertyAurora}

\begin{propertyAurora}{Tilefade}
This property is only relevant for tiles. It controls whether this 
mesh will turn invisible to clear the view.
\begin{description}[leftmargin=6em,style=nextline]
    \item[None] This mesh will always be visible.
    \item[Fade] The object will fade.
    \item[Base] Unknown
    \item[Neighbour] The object will fade along with meshes in neighboring tiles.
\end{description}
\end{propertyAurora}

\begin{propertyAurora}{Render}
This property controls whether this object should be rendered. Note: Meshes can still
cast a shadow without being rendered.
\end{propertyAurora}

\begin{propertyAurora}{Shadow}
Controls whether this objects should cast a shadow. It is recommended to 
disable Shadows for complex Meshes with a large amount of triangles and 
use a create a lower poly mesh with disabled rendering and enabled shadows.

Failing to do so might negatively impact performance and corrupt the shadows of the 
model as the engine is only capable of handling shadows for simple, convex objects.
\end{propertyAurora}

\begin{propertyAurora}{Beaming}
Unknown. Probably unused.
\end{propertyAurora}

\begin{propertyAurora}{Inherit Color}
Unknown. Probably unused.
\end{propertyAurora}

\begin{propertyAurora}{Rotatetexture}
Only available for tiles. Auto-Rotates textures so the UVs are rotated
the same way to avoids seams between tiles.
\end{propertyAurora}

\begin{propertyAurora}{Transparencyhint}
This helps the engine prioritize transparency, similar to a z-buffer.
Multiple overlapping transparent meshes can cause issues like
flickering, which may be resolved by changing this property. Objects 
with lower transparencyhint values are rendered first.\\

This property can be ignored if neither material nor texture have transparency.
\end{propertyAurora}

\begin{propertyAurora}{Smoothgroup}
Controls smoothing groups or shading groups are created
\begin{description}[leftmargin=6em,style=nextline]
    \item[Separate] Each face will have its own smoothing group. This results in no smoothing at all.
    \item[Single] All faces belong to a single smoothing group. Meshes will be smoothed completely.
    \item[Auto] Auto generate smoothing groups, depending on the settings in blender and edges marked as sharp. Replicates blenders smoothing as closely as possible.
\end{description}
\end{propertyAurora}

\subsection{Danglymeshes}
Danglymeshes are Trimeshes which are "bouncy" or "dangly". They are affected by
wind, character movement and visual effects. Danglymeshes possess the same 
properties as Trimeshes in addition to the following ones.

\begin{propertyAurora}{Dangle group} 
The dangle group is a vertex group containing the weights of vertices for the danglymesh. 
You must select an existing vertex group. Vertex paint may be used to paint the weights. A vertex with a weight of
0.0 will not change position.
\end{propertyAurora}

\begin{propertyAurora}{Period} 
Determines how quickly the vertices move and how fast they come to a rest.
\end{propertyAurora}
\begin{propertyAurora}{Tightness} 
The 'tension' in the object, restricting movement. 
\end{propertyAurora}
\begin{propertyAurora}{Displacement} 
Determine show far the vertices are allowed to move. This is further restricted by the Dangle group.
\end{propertyAurora}


\subsection{Skinmeshes}
A skinmesh is used to create skeletal animations. At the moment it is not 
possible to use Blender's armature system directly. Instead several vertex groups 
must be created with their names matching one of the other trimeshes in the MDL. 
Unfortunately this means that no animations are visible in blender which makes
creation of new skinmeshes difficult.


\subsection{Animmeshes}
Animeshes are Trimeshes with animated UV coordinates or vertices. The latter is unsupported. 
It is recommended to use the AnimAll Add-On to enable insertion of keyframes 
for UV coordinates. The number of animated keyframes for Animmeshes is limited to 
two (start and beginning of 
the animation).


\subsection{Walkmeshes}
Walkmeshes indicate where a character can walk. It is necessary to select 
the appropriate type of walkmesh, which depends on the type of model itself. 

\begin{propertyAurora}{Walkmeshtype} 
Sets the type of a walkmesh. The available types depends on the classification of the MDL:
\begin{description}[leftmargin=10em,style=nextline]
    \item[Tileset] Walkmesh for tiles. This will also affect footstep sounds and grass growth. No overlapping faces are allowed. To specify the surface type, it is necessary to add the walkmesh materials to the object. You can add all walkmesh materials by clicking on the {\textit{Load walkmesh materials}} Button in the Aurora mesh properties panel. The materials are then added as materials slots and accessible through the materials tab in the object properties.
    The export script will create a matching {\textit{.wok}} file. 
    \item[Door: Closed] The walkmesh for the closed state of a door. Blocks movement.
    \item[Door: Open 1] The Walkmesh for the first open state of a door. Blocks movement.
    \item[Door: Open 2] The Walkmesh for the second open state of a door. Blocks movement.
    \item[Placeable] Walkmesh for placeables. Blocks movement.
\end{description}
\end{propertyAurora}

\section{Emitters}
Support for Emitters is rudimentary. Blenders particle systems is too
different to allow a direct import, instead Emitters are imported as plain
text. While all data will be retained, editing is only possible by altering the 
generated text file for each emitter.

\section{Lamps}
Blender's lamp properties will be used for exporting a light.
Exported properties are:
\begin{itemize}
\item Color
\item Distance
\end{itemize}
It is recommended to check \textit{Sphere} to better replicate the resulting effect 
of a light source in the 3D view.

\begin{propertyAurora}{Light type}
There are some special light types, all of which are used to give builders
the ability to select light color in the Toolset.
\begin{description}[leftmargin=8em,style=nextline]
    \item[Mainlight 1] For Tiles only. The Lamp Color property will not affect this lamp as it can be changed in the Toolset.
    \item[Mainlight 2] For Tiles only. The Lamp Color property will not affect this lamp as it can be changed in the Toolset.
    \item[Sourcelight 1] For Tiles only. This is actually a burning flame. Color can be changed in the Toolset.
    \item[Sourcelight 2] For Tiles only. This is actually a burning flame. Color can be changed in the Toolset.
    \item[Default] Default type, can always be used. Blenders lamp properties will be used (color and distance)
\end{description}
\end{propertyAurora}

\begin{propertyAurora}{Priority}
Unknown. Might control when the light source casts a shadow.
\end{propertyAurora}

\begin{propertyAurora}{Ambient Only}
This controls if the light is only an ambient light source or
if it is directional as well.
\end{propertyAurora}

\begin{propertyAurora}{Fading}
Unknown. Might activate some kind of distance fall off for the light.
\end{propertyAurora}

\begin{propertyAurora}{Shadows}
Determines if this light is capable of producing shadows.
\end{propertyAurora}

\begin{propertyAurora}{Is Dynamic}
Unknown.
\end{propertyAurora}

\begin{propertyAurora}{Affect Dynamic}
This controls whether this light affects dynamic objects, i.e. characters.
Disabling this will prevent this light from producing shadows with dynamic
objects, but it will in turn improve performance.

This is a less strict version of the \textit{Shadows} setting.
\end{propertyAurora}

\begin{propertyAurora}{Lensflares}
Add a series of lens-flares for this light.

Lens-flares are unavailable for Mainlights or Sourcelights.
\begin{description}[leftmargin=6em,style=nextline]
    \item[Texture] Texture for this flare
    \item[Colorshift] Unknown. Possibly color difference from the light color.
    \item[Size] Size of the flares. This will scale the texture.
    \item[Position] Distance from the lights origin.
\end{description}
\end{propertyAurora}

\section{Materials \& Textures}
MDL files support a single material with a single texture. If multiple 
materials or textures are assigned, the export script will choose the currently 
active one or, failing that, the topmost one.

The following properties are exported:
\begin{itemize}
    \item Diffuse
    \item Specular
    \item Alpha values
\end{itemize}

\subsection*{Alpha}
Neverblender can export either material or texture alpha. When both properties are present 
the latter takes precedence. This matches the output in blenders 3D view (with GLSL mode enabled) 
or renderer.

\section{Animations}
Each animation has to be defined by name and start/end frames. Valid names 
are listed in the reference chapter for each type of MDL.

The following properties can be animated:
\begin{itemize}
    \item Location
    \item Rotation
    \item Scale (Uniform only)
    \item Material or Texture Alpha
    \item Self-Illumination color
    \item Lamp color
    \item Lamp distance (Light radius)
\end{itemize}
Animation data like start/end frames and name are stored in the Rootdummy. Deleting the Rootdummy will result in
the loss of this data.
\section{Editing}
There are three types of objects which can be exported as mdl: Empties,
Meshes and Lamps. Other objects like curves and surfaces will have to be
converted to meshes before attempting to export them.

\subsection{Empties}
Empties appear in mdl files as Dummies. Like in blender they are used to group
objects together. In addition there are special types of Dummies which for
example indicate locations for spells effects.

\subsubsection{Dummy}

\begin{description}
    \item[Ground] TODO
    \item[Impact] TODO
    \item[Head Hit] TODO
    \item[Head] TODO
    \item[Hand] TODO
    \item[Use 1] TODO
    \item[Use 2] TODO
    \item[None] TODO
\end{description}
At the very least placeables should have an Impact, Use 1 and Use 2 dummys.
While they are not strictly required, they will make sure that spells
work correctly and make them usable by characters.

\subsubsection{Rootdummy}
All objects belonging to a mdl have to be parented to a single Empty, which
has no parent itself. This Empty must have the same name as the mdl file
(minus the file extension).

It holds additional information about the model.
\begin{description}
    \item[Classification] TODO
    \item[Supermodel] References another mdl file. All Animations will be added to this mdl.
    \item[Animationscale] Unknown
\end{description}
It is also used to store the beginning and end frames of specific animations
(walk, run, sit,\ldots).

\subsubsection{Reference Node}
The purpose and usage of these nodes is unknown. Supposedly it can be used to
reference other mdl files. Reference nodes can be found in some spells and
most of the time references fx\_ref.

\subsubsection{Patch Node}
The purpose and usage of these nodes is unknown. They occur in some spells, but
they seem to behave excatly like normal dummy nodes, i.e they have the same
attributes.

\subsection{Meshes}

\subsubsection{Trimeshes}

\begin{description}
    \item[Wirecolor]
    \item[Self-illumination color]
    \item[Ambient Color]
    \item[Shininess]
    \item[Tilefade]
    \item[Render]
    \item[Shadow]
    \item[Beaming]
    \item[Inherit Color]
    \item[Rotate Texture]
    \item[Transparency Hint]
    \item[Smoothgroup]
\end{description}

\subsubsection{Danglymeshes}
\begin{description}
    \item[Dangle group] The dangle group is a vertex group containing the weights of vertices for the danglymesh. You must select an existing vertex group.
    \item[Period]
    \item[Tightness]
    \item[Displacement]
\end{description}

\subsubsection{Skinmeshes}

\subsubsection{Walkmeshes}
\begin{description}
    \item[Tileset] Walkmesh for tiles. Use Materials to denote surface type of
                   tile (stone, grass, ...). This will also affect footstep
                   sound and grass growth.
    \item[Door: Closed] The walkmesh for the closed state of the door
    \item[Door: Open 1] The Walkmesh for the first open state of the door
    \item[Door: Open 2] The Walkmesh for the second open state of the door
    \item[Placeable]
\end{description}
\subsection{Emitters}

\subsection{Lamps}

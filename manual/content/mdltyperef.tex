\chapter{Mdl types}

\section{Creatures}
TODO

\section{Doors}
Placeable should have at least the following dummies:
\begin{itemize}
\item Use1
\item Use2
\item Impact
\end{itemize}
They will ensure it can be used properly and spell/effects hit the right spot.

\subsection*{Possible Animations}
\begin{description}
    \item[closed] Closed state of the door.
    \item[open1] First open state of the door. Doors in NWN can open/swing to both sides, depending from which side a player is opening it. Open 1 and Open 2 can be the same.
    \item[opening1] Transition from closed to open1
    \item[closing1] Transition from open1 to closed.
    \item[open2] Second open state of the door. Doors in NWN can open/swing to both sides, depending from which side a player is opening it. Open 1 and Open 2 can be the same.
    \item[opening2] Transition from closed to open2.
    \item[closing2] Transition from open2 to closed.
    \item[trans] Unknown
    \item[die] Animation when the placeable is destroyed. Most placeables omit this alltogether or sink into the ground.
    \item[dead] Dead state. Most placeables omit this alltogether or have a single positionkey below ground level.
\end{description}

\section{Effects}
TODO


\section{Placeables}

\subsection*{Possible Animations}
\begin{description}
    \item[default] Default state. It is recommended to add this if there are other animations present. If this is omitted and other animations are present a placeable can't be set to static in the toolset (else the game will crash)
    \item[on]
    \item[on2off] Transition from on to off state.
    \item[off]
    \item[off2on] Transition from off to on state.
    \item[close] Closed state.
    \item[close2open] Transition from closed to open state.
    \item[open] Open state.
    \item[open2close] Transition from open to closed state.
    \item[damage] Animation when the placeable is damaged. Most placeables omit this animation or willshake for 0.1 seconds.
    \item[die] Animation when the placeable is destroyed. Most placeables omit this alltogether or sink into the ground.
    \item[dead] Dead state. Most placeables omit this alltogether or have a single positionkey below ground level.
\end{description}

\section{Tiles}

\subsection*{Tile Walkmesh (AABB)}
The walkmesh must be parented to the the {\textit{Rootdummy}} object.
No overlapping faces are allowed. To specify the surface type, it is necessary
to add the walkmesh materials to the object. You can add all walkmesh
materials by clicking on the {\textit{Load walkmesh materials}} Button in the
Aurora mesh properties panel. \\ \\

The walkmesh materials are then added as materials slots and accessible
through the materials tab in the object properties. \\ \\

The export script will create an aabb tree from the walkmesh during export
and add it to the {\textit{*.mdl}} file. It will also create a
matching {\textit{*.wok}} file.


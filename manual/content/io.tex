\chapter{Import \& Export}

\section{Import}
\begin{enumerate}
	\item Go to \textit{File >> Import > Aurora (*.mdl)}
	\item In the file dialog you find some options on the left. Scroll down to the panel if necessary. The available options are explained blow.
	\item Press the button \textit{Import Aurora MDL}
\end{enumerate}

\begin{description}[leftmargin=13em,style=nextline]
    \item[Import Geometry] Import the geometry from the mdl file.
    \item[Import Walkmesh] Attempt to import a walkmesh. If the imported model is a placeable, the script will look for a {\textit{pwk}} file in the same folder. If the model is a door, it will look for a {\textit{dwk}} file. If the model is a tile, it will read the walkmesh directly from the {\textit{mdl}} file.
    \item[Import Smooth Groups] Import smoothing groups as sharp edges.
    \item[Import Animations] Import animation from the mdl file. Animations are added to the imported geometry. If no geometry has benn imported, the script will try to add animations to already existing objects in blender.
    \item[Materials] None = No materials or textures will be imported. Single = Neverblender will attempt to merge similar materials to reduce clutter. Multiple = Each object will get it's own material, even if this results in multiple identical materials.
    \item[Image Search] Search for textures in sub-directories. This might take a significant amount of time depending on the number of files.
\end{description}

\section{Export}
\begin{enumerate}
	\item Go to \textit{File >> Export > Aurora (*.mdl)}
	\item In the file dialog you find some options on the left. Scroll down to the panel if necessary. The available options are explained blow.
	\item The filename has be the same as the name of the Rootdummy.
\end{enumerate}

\begin{description}[leftmargin=12em,style=nextline]
    \item[Export Animations] Export animations to mdl.
    \item[Export Walkmesh] Attempt to export a walkmesh. The type of the exported walkmesh depends on the objects classification.
    \item[Export Smooth Groups] Generate smoothing groups. Method depends on the settings of each mesh.
    \item[Apply Modifiers] Apply modifiers before exporting.
\end{description}

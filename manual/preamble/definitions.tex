%-------------------------------------------------------------------------------
%	DEFINITION OF COLORED BOXES
%-------------------------------------------------------------------------------

\newmdenv[skipabove=7pt,
          skipbelow=7pt,
          backgroundcolor=black!5,
          linecolor=black!50,
          topline=false,
          rightline=false,
          bottomline=false,
          leftline=true,
          innertopmargin=5pt,
          innerrightmargin=5pt,
          innerbottommargin=5pt,
          innerleftmargin=5pt,
          leftmargin=0cm,
          rightmargin=0cm,
          linewidth=4pt]{mdeBlenderProperty}

\newmdenv[skipabove=7pt,
          skipbelow=7pt,
          backgroundcolor=black!5,
          linecolor=maincolor,
          topline=false,
          rightline=false,
          bottomline=false,
          leftline=true,
          innertopmargin=5pt,
          innerrightmargin=5pt,
          innerbottommargin=5pt,
          innerleftmargin=5pt,
          leftmargin=0cm,
          rightmargin=0cm,
          linewidth=4pt]{mdeAuroraProperty}

%-------------------------------------------------------------------------------
%	REMARK ENVIRONMENT
%-------------------------------------------------------------------------------

\newenvironment{remark}
{\par\vspace{10pt}\small % Vertical white space above the remark and smaller font size
    \begin{list}{}{\leftmargin=35pt % Indentation on the left
            \rightmargin=25pt}\item\ignorespaces % Indentation on the right
        \makebox[-2.5pt]{\begin{tikzpicture}[overlay]
            \node[draw=maincolor!60,line width=1pt,circle,fill=maincolor!25,font=\sffamily\bfseries,inner sep=2pt,outer sep=0pt] at (-15pt,0pt){\textcolor{maincolor}{R}};\end{tikzpicture}} % Orange R in a circle
        \advance\baselineskip -1pt}
    {\end{list}\vskip5pt} % Tighter line spacing and white space after remark

%-------------------------------------------------------------------------------
%	PROPERTY ENVIRONMENTS
%-------------------------------------------------------------------------------

\newenvironment{propertyBlender}[1]
    {\paragraph{{\textcolor{gray}{\rule{0.5\baselineskip}{0.5\baselineskip}}}~~#1}\index{#1}}
    {}

\newenvironment{propertyBlenderOld}[1]
    {\begin{mdeBlenderProperty}{\sffamily\bfseries#1}\index{#1}\\}
    {\end{mdeBlenderProperty}}


\newenvironment{propertyAurora}[1]
    {\paragraph{{\textcolor{maincolor}{\rule{0.5\baselineskip}{0.5\baselineskip}}}~~#1}\index{#1}}
    {}

\newenvironment{propertyAuroraOld}[1]
    {\begin{mdeAuroraProperty}{\sffamily\bfseries#1}\index{#1}\\}
    {\end{mdeAuroraProperty}}

%-------------------------------------------------------------------------------
%	PAGE HEADERS
%-------------------------------------------------------------------------------

\pagestyle{fancy}
\renewcommand{\chaptermark}[1]{\markboth{\sffamily\normalsize\bfseries\chaptername\ \thechapter.\ #1}{}} % Chapter text font settings
\renewcommand{\sectionmark}[1]{\markright{\sffamily\normalsize\thesection\hspace{5pt}#1}{}} % Section text font settings
\fancyhf{} \fancyhead[LE,RO]{\sffamily\normalsize\thepage} % Font setting for the page number in the header
\fancyhead[LO]{\rightmark} % Print the nearest section name on the left side of odd pages
\fancyhead[RE]{\leftmark} % Print the current chapter name on the right side of even pages
\renewcommand{\headrulewidth}{0.5pt} % Width of the rule under the header
\addtolength{\headheight}{2.5pt} % Increase the spacing around the header slightly
\renewcommand{\footrulewidth}{0pt} % Removes the rule in the footer
\fancypagestyle{plain}{\fancyhead{}\renewcommand{\headrulewidth}{0pt}} % Style for when a plain pagestyle is specified

% Removes the header from odd empty pages at the end of chapters
\makeatletter
\renewcommand{\cleardoublepage}{
    \clearpage\ifodd\c@page\else
    \hbox{}
    \vspace*{\fill}
    \thispagestyle{empty}
    \newpage
    \fi}
\makeatother

%-------------------------------------------------------------------------------
%	SECTION NUMBERING IN THE MARGIN
%-------------------------------------------------------------------------------

\makeatletter
\renewcommand{\@seccntformat}[1]{\llap{\textcolor{maincolor}{\csname the#1\endcsname}\hspace{1em}}}                    
\renewcommand{\section}{\@startsection{section}{1}{\z@}
    {-4ex \@plus -1ex \@minus -.4ex}
    {1ex \@plus.2ex }
    {\normalfont\large\sffamily\bfseries}}
\renewcommand{\subsection}{\@startsection {subsection}{2}{\z@}
    {-3ex \@plus -0.1ex \@minus -.4ex}
    {0.5ex \@plus.2ex }
    {\normalfont\sffamily\bfseries}}
\renewcommand{\subsubsection}{\@startsection {subsubsection}{3}{\z@}
    {-2ex \@plus -0.1ex \@minus -.2ex}
    {.2ex \@plus.2ex }
    {\normalfont\small\sffamily\bfseries}}                        
\renewcommand\paragraph{\@startsection{paragraph}{4}{\z@}
    {-2ex \@plus-.2ex \@minus .2ex}
    {.1ex}
    {\normalfont\small\sffamily\bfseries}}

%-------------------------------------------------------------------------------
%	PART HEADINGS
%-------------------------------------------------------------------------------

% numbered part in the table of contents
\newcommand{\@mypartnumtocformat}[2]{%
    \setlength\fboxsep{0pt}%
    \noindent\colorbox{maincolor!20}{\strut\parbox[c][.7cm]{\ecart}{\color{maincolor!70}\Large\sffamily\bfseries\centering#1}}\hskip\esp\colorbox{maincolor!40}{\strut\parbox[c][.7cm]{\linewidth-\ecart-\esp}{\Large\sffamily\centering#2}}}%
%%%%%%%%%%%%%%%%%%%%%%%%%%%%%%%%%%
% unnumbered part in the table of contents
\newcommand{\@myparttocformat}[1]{%
    \setlength\fboxsep{0pt}%
    \noindent\colorbox{maincolor!40}{\strut\parbox[c][.7cm]{\linewidth}{\Large\sffamily\centering#1}}}%
%%%%%%%%%%%%%%%%%%%%%%%%%%%%%%%%%%
\newlength\esp
\setlength\esp{4pt}
\newlength\ecart
\setlength\ecart{1.2cm-\esp}
\newcommand{\thepartimage}{}%
\newcommand{\partimage}[1]{\renewcommand{\thepartimage}{#1}}%
\def\@part[#1]#2{%
    \ifnum \c@secnumdepth >-2\relax%
    \refstepcounter{part}%
    \addcontentsline{toc}{part}{\texorpdfstring{\protect\@mypartnumtocformat{\thepart}{#1}}{\partname~\thepart\ ---\ #1}}
    \else%
    \addcontentsline{toc}{part}{\texorpdfstring{\protect\@myparttocformat{#1}}{#1}}%
    \fi%
    \startcontents%
    \markboth{}{}%
    {\thispagestyle{empty}%
        \begin{tikzpicture}[remember picture,overlay]%
        \node at (current page.north west){\begin{tikzpicture}[remember picture,overlay]%	
            \fill[maincolor!20](0cm,0cm) rectangle (\paperwidth,-\paperheight);
            \node[anchor=north] at (4cm,-3.25cm){\color{maincolor!40}\fontsize{220}{100}\sffamily\bfseries\@Roman\c@part}; 
            \node[anchor=south east] at (\paperwidth-1cm,-\paperheight+1cm){\parbox[t][][t]{8.5cm}{
                    \printcontents{l}{0}{\setcounter{tocdepth}{1}}%
                }};
                \node[anchor=north east] at (\paperwidth-1.5cm,-3.25cm){\parbox[t][][t]{15cm}{\strut\raggedleft\color{white}\fontsize{30}{30}\sffamily\bfseries#2}};
                \end{tikzpicture}};
        \end{tikzpicture}}%
        \@endpart}
    \def\@spart#1{%
        \startcontents%
        \phantomsection
        {\thispagestyle{empty}%
            \begin{tikzpicture}[remember picture,overlay]%
            \node at (current page.north west){\begin{tikzpicture}[remember picture,overlay]%	
                \fill[maincolor!20](0cm,0cm) rectangle (\paperwidth,-\paperheight);
                \node[anchor=north east] at (\paperwidth-1.5cm,-3.25cm){\parbox[t][][t]{15cm}{\strut\raggedleft\color{white}\fontsize{30}{30}\sffamily\bfseries#1}};
                \end{tikzpicture}};
        \end{tikzpicture}}
        \addcontentsline{toc}{part}{\texorpdfstring{%
                \setlength\fboxsep{0pt}%
                \noindent\protect\colorbox{maincolor!40}{\strut\protect\parbox[c][.7cm]{\linewidth}{\Large\sffamily\protect\centering #1\quad\mbox{}}}}{#1}}%
        \@endpart}
    \def\@endpart{\vfil\newpage
        \if@twoside
        \if@openright
        \null
        \thispagestyle{empty}%
        \newpage
        \fi
        \fi
        \if@tempswa
        \twocolumn
        \fi}
    
%-------------------------------------------------------------------------------
%	CHAPTER HEADINGS
%-------------------------------------------------------------------------------

\def\@makechapterhead#1{%
    {\parindent \z@ \raggedright \normalfont
        \ifnum \c@secnumdepth >\m@ne
        \if@mainmatter
        \begin{tikzpicture}[remember picture,overlay]
        \node at (current page.north west)
        {\begin{tikzpicture}[remember picture,overlay]
            \node[anchor=north west,inner sep=0pt] at (0,0) {\includegraphics[width=\paperwidth]{\chapterimage}};
            \draw[anchor=west] (\Gm@lmargin,-6cm) node [line width=2pt,rounded corners=15pt,draw=maincolor,fill=white,fill opacity=0.5,inner sep=15pt]{\strut\makebox[22cm]{}};
            \draw[anchor=west] (\Gm@lmargin+.3cm,-6cm) node {\huge\sffamily\bfseries\color{black}\thechapter. #1\strut};
            \end{tikzpicture}};
    \end{tikzpicture}
    \else
    \begin{tikzpicture}[remember picture,overlay]
    \node at (current page.north west)
    {\begin{tikzpicture}[remember picture,overlay]
        \node[anchor=north west,inner sep=0pt] at (0,0) {\includegraphics[width=\paperwidth]{\chapterimage}};
        \draw[anchor=west] (\Gm@lmargin,-6cm) node [line width=2pt,rounded corners=15pt,draw=maincolor,fill=white,fill opacity=0.5,inner sep=15pt]{\strut\makebox[22cm]{}};
        \draw[anchor=west] (\Gm@lmargin+.3cm,-6cm) node {\huge\sffamily\bfseries\color{black}#1\strut};
        \end{tikzpicture}};
\end{tikzpicture}
\fi\fi\par\vspace*{180\p@}}}

%-------------------------------------------

\def\@makeschapterhead#1{%
    \begin{tikzpicture}[remember picture,overlay]
        \node at (current page.north west){
            \begin{tikzpicture}[remember picture,overlay]
                \node[anchor=north west,inner sep=0pt] at (0,0) {\includegraphics[width=\paperwidth]{\chapterimage}};
                \draw[anchor=west] (\Gm@lmargin,-6cm) node [line width=2pt,rounded corners=15pt,draw=maincolor,fill=white,fill opacity=0.5,inner sep=15pt]{\strut\makebox[22cm]{}};
                \draw[anchor=west] (\Gm@lmargin+.3cm,-6cm) node {\huge\sffamily\bfseries\color{black}#1\strut};
            \end{tikzpicture}};
    \end{tikzpicture}
    \par\vspace*{180\p@}}
\makeatother
